
% setup.tex

% Codifica dei caratteri e lingua
\usepackage[T1]{fontenc}
\usepackage{newlfont}
\usepackage[italian]{babel}

% Margini e formattazione della pagina
\usepackage{geometry} %[a4paper, top=2.5cm, bottom=2.5cm, left=3cm, right=3cm, bindingoffset=0cm]

% \usepackage{amssymb}
% \usepackage{latexsym}
% \usepackage{mathtools}
\usepackage{mathptmx}

% Pacchetto per l'inclusione di immagini
\usepackage{graphicx}
\graphicspath{ {../img/} }

% Pacchetto per la gestione delle unità di misura
\usepackage{siunitx}

% Pacchetto per l'impaginazione avanzata
\usepackage{fancyhdr}

% Pacchetto per la gestione avanzata delle tabelle
\usepackage{tabularx}

% Pacchetto per la creazione di liste personalizzate
\usepackage{enumitem}


% Pacchetto per la creazione di collegamenti ipertestuali
\usepackage{amsthm}
\usepackage{hyperref}
\hypersetup{
    colorlinks=true,
    linkcolor=black,
    citecolor=green,
    filecolor=magenta,
    urlcolor=cyan,
}
\usepackage{cleveref}

\usepackage{indentfirst}%%libreria per avere l'indentazione all'inizio dei capitoli, ...

% Pacchetto per la formattazione avanzata delle didascalie delle figure e delle tabelle
\usepackage[font=small, labelfont=bf, labelsep=period]{caption}

% Pacchetto per l'uso di colori
\usepackage{xcolor}

% Pacchetto per la formattazione avanzata delle pagine dei titoli dei capitoli
\usepackage{titlesec}

% Pacchetto per la creazione di elenchi puntati personalizzati
\usepackage{enumitem}

% Altri pacchetti personalizzati e configurazioni possono essere aggiunti qui

% Configurazioni personalizzate
% Esempio: Cambia il formato dei titoli dei capitoli
%\titleformat{\chapter}[display]{\normalfont\huge\bfseries}{\chaptertitlename\ \thechapter}{20pt}{\Huge}
%\usepackage{afterpage}
\newcommand\blankpage{%
    \null
    \thispagestyle{empty}%
    \addtocounter{page}{-1}%
    \newpage}
\usepackage[sorting=none]{biblatex}
\addbibresource{setup/references.bib}

% Fine del file setup.tex

% Old files
% \usepackage[colorinlistoftodos]{todonotes}
% %\usepackage{ulem}
% \usepackage{graphicx}%libreria per inserire grafici
% \graphicspath{ {../img/} }
% %%librerie matematiche


% \usepackage{hyperref}
% \hypersetup{
%     colorlinks=false,
%     linkcolor=blue,
%     filecolor=magenta,      
%     urlcolor=cyan,
% }
% \usepackage{fancyhdr} %%libreria per impostare il documento


% \usepackage{cleveref}

% \usepackage{listings}
% \usepackage{booktabs}
% \usepackage{subfig}
% \usepackage{float}

% \usepackage{multicol}

% \usepackage{adjustbox}


% \usepackage{vmargin} 
% \oddsidemargin=30pt \evensidemargin=20pt%impostano i margini
% \hyphenation{sil-la-ba-zio-ne pa-ren-te-si}%serve per la 
% \hypersetup{
%     colorlinks,
%     citecolor=black,
%     filecolor=black,
%     linkcolor=black,
%     urlcolor=black
% }
% \usepackage{enumerate}% http://ctan.org/pkg/enumerate
% \lstloadlanguages{bash}%\titleformat{\chapter}[hang]{\Huge\bfseries}{\thechapter\hsp\textcolor{gray75}{|}\hsp}{0pt}{\Huge\bfseries}
% %\lstinputlisting{NOME_FILE.ESTENSIONE} inserire direttamente il file
% %Dloating Figures
% \usepackage{wrapfig}

% %\geometry{bottom=3cm}
% %Times new roman

% %Interlinea
%  \linespread{1.3}                        %comando per impostare l'interlinea

% \renewcommand{\baselinestretch}{1.3} 


% % For images---------------------------------------

% % If you use the hyperref package, please uncomment the following line
% % to display URLs in blue roman font according to Springer's eBook style:
% %\renewcommand\UrlFont{\color{blue}\rmfamily}

% % New commands
% %\newcommand{\NDA}[1]{\todo[inline,color=cyan]{\textit{[Andrea: #1]}}}
% %\newcommand{\NDM}[1]{\todo[inline,color=red]{\textit{[Martin: #2]}}}

% % \hyphenation{com-pu-ta-tion-ally com-pu-ta-ti-on-ally-in-ten-si-ve}
