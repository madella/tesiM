
\thispagestyle{empty}
\setmarginsrb{3 cm}{2 cm}{3 cm}{0 cm}{-1 cm}{1 cm}{0 cm}{1.5 cm}
\begin{center}
    % \centering
    %\includegraphics[width=0.35\textwidth]{Immagini/Senza titolo-1.png}%LOGO 
    
	\noindent\LARGE
    UNIVERSITÀ DEGLI STUDI DI MODENA E REGGIO EMILIA\\[0.7cm]
    \Large
    Dipartimento di Ingegneria “Enzo Ferrari”\\[0.8cm]
	\Large
	Corso di Laurea in Ingegneria Informatica \\[1.8cm]
	\LARGE
	{\bfseries Docker e Kubernetes nell'ambito del Cloud Computing}\\[1.2cm]
\end{center}
\vspace{4.5cm}
%\vfill
\begin{minipage}{0.35\textwidth}
{\bfseries Relatore}\\
Prof. Francesco Guerra   
\end{minipage}
\hfill
\begin{minipage}{0.35\textwidth}
\begin{flushright} {\bfseries Candidato}\\
Giacomo Madella \end{flushright} 
\end{minipage} \\
\vfill
\begin{center}
	Aprile 2021
\end{center}

\afterpage{\blankpage}



% %\title{A secure and distributed middleware for green computing} %
% \title{An interoperable runtime for distributed power management of large-scale HPC systems based on DDS.}
% % TITLE : sullo sviluppo di un run-time per il power management basato su DDS, astratto all’interno di una libreria simil ROS2
% \titlerunning{An interoperable runtime for green computing}
% \author{Giacomo Madella}
% \authorrunning{Madella}
% \institute{Alma Mater Studiorum, Università di Bologna \\
% Research Topic: Piattaforme distribuite, sicure e interoperabili per l'efficientamento dei sistemi di calcolo larga scala}
% \date{July 11, 2023}
% \maketitle              % typeset the header of the contribution

