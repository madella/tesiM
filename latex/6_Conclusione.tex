\chapter{Conclusioni}

È stato dimostrato come un framework di comunicazione DDS può essere usato all'interno di un Power-Stack per la gestione di energia in sistemi HPC vincolati dalla potenza al fine di affrontare il problema della limitazione energetica. %Con questo framework, è possibile specificare e calibrare facilmente l'hardware del sistema. Nel frattempo, è utile anche per compiti come la regolazione delle applicazioni utente per massimizzare le prestazioni o ridurre la domanda di energia. Per verificare la validità e l'utilità del framework, è stato testato con diversi casi di studio. In questi casi di studio, è stata applicata la gestione dell'energia a due applicazioni selezionate e è stata dimostrata la possibilità di costruire e utilizzare un semplice modello di potenza con una relazione lineare tra le prestazioni della CPU e il consumo di energia per derivare il limite di potenza. Questi casi di studio hanno semplicemente dimostrato che il framework può fornire agli utenti un modo semplice per applicare l'ottimizzazione e la gestione dell'energia alle loro applicazioni. Nel futuro lavoro, è previsto valutare il framework proposto con altre politiche/algoritmi di ottimizzazione delle prestazioni e dell'energia e migliorarlo con ulteriori funzionalità come la cooperazione con il software di sistema, i pianificatori di lavori e altri strumenti esterni per arricchire le sue funzionalità.