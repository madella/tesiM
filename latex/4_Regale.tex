\chapter{REGALE}\label{chap:4_REGALE}
%COSA
REGALE\cite{REGALE} è un progetto finanziato dall'UE\cite{ue_REGALE} nato ad Aprile 2021 che opera nell'ambito del Power Management in sistemi ad High-Performance Computing ed in particolare si è focalizzato su sistemi Exascale\footnote{Exascale: capace di eseguire operazioni nell'ordine di ExaFlops ($10^{18}$)}.

\section{Obbiettivi}
Il loro principale obbiettivo è quello di aprire la strada alla prossima generazione di applicazioni per HPC, riunendo accademici e centri europei di supercalcolo. Il progetto si pone di definire un'architettura open-source con l'intenzione di costruire un prototipo in grado di dotare i sistemi di HPC dei meccanismi e delle politiche necessari per garantire un utilizzo delle risorse efficace\cite{ue_REGALE}. Per farlo sono inoltre state definite delle politiche da seguire durante lo sviluppo di tutto il progetto:
\begin{itemize}
    \item Effettivo utilizzo delle risorse disponibili, tramite aumento del throughput del sistema e la minimizzazione della \emph{Performance Degradation} sotto vincoli di potenza;
    \item Ampia applicabilità attraverso l'inseguimento di concetti come scalabilità, indipendenza dalle piattaforme ed estensibilità;
    \item Facilità di implementazione tramite la creazione di una infrastruttura flessibile, e che gestisca in automatico le risorse.
\end{itemize}


\section{Power Stack}
L'intero progetto di Power Stack, durante il suo sviluppo, si è basato su strumenti come MPI library\cite{mpi}, SLURM\cite{slurm}, e DCDB\cite{dcdb}. Inoltre, è stato deciso di introdurre software open-source che potessero soddisfare le esigenze del modello di Power Stack~\ref{fig:powerstackscheme}. Infatti sono stati valutati e selezionati diversi applicativi (molti dei quali prodotti dai partner, come mostrati in tabella~\ref{table:REGALE}) anche con ruoli analoghi, per soddisfare diverse esigenze.
\begin{table}[ht]
    \centering
    \begin{tabular}{l|l|l}
    \hline
    \textbf{Tool} & \textbf{Partner} & \textbf{Ruolo all'interno di REGALE} \\
    \hline
    SLURM & TUM & System Manager \\
    \hline
    OAR & UGA & System Manager \\
    \hline
    DCDB & LRZ & Monitor, Monitoring Data \\
    \hline
    BEO & ATOS & Monitor, Node Manager, Monitoring Data \\
    \hline
    BDBO & ATOS & Monitor, Job Manager \\
    \hline
    EAR & BSC & Monitor, Node Manager, Job Manager, Monitoring Data \\
    \hline
    Melissa & UGA & Workflow Engine \\
    \hline
    RYAX & RYAX & Workflow Engine \\
    \hline
    Examon & E4/UNIBO & Monitor, Monitoring Data \\
    \hline
    COUNTDOWN & CINECA/UNIBO & Job Manager \\
    \hline
    PULPcontroller & UNIBO & Node Manager \\
    \hline
    BeBiDa & RYAX & System Manager \\
    \hline
\end{tabular}
\caption{Software introdotti all'interno di REGALE con il partner che li ha prodotti e il loro ruolo}\label{table:REGALE}
\end{table}

\begin{figure}[H]
    \centering
    \begin{subfigure}[b]{0.50\textwidth}
    \includegraphics[width=\textwidth]{img/REGALE-Architecture-1536x1421.png}
    \caption{Modello di Power Stack Regale}\label{fig:powerstackscheme}
    \end{subfigure}
    \hfill
    \begin{subfigure}[b]{0.50\textwidth}
    \includegraphics[width=\textwidth]{img/REGALE_USE.png}
    \caption{Copertura componenti}\label{fig:regale_cover}
    \end{subfigure}
    \caption{Implementazione dei componenti secondo il modello del Power stack}
\end{figure}
Vista la natura dei software introdotti nel progetto, non era previsto che questi potessero scambiare informazioni tra di loro, in quanto nati per essere usati singolarmente. Per questo, e per la difficoltà di creare interfacce di comunicazione specifiche per ogni coppia di componente, si è scelto di procedere con un \textbf{middleware DDS} unificato. In questa decisione è nata la necessità di sperimentare e scegliere una implementazione di DDS adeguata al sistema di riferimento. %Questa tesi è nata in collaborazione con REGALE, ed è stata stilata anche per riportare test utili alla finalità di REGALE.