\chapter{Conclusioni}

Nel primo capitolo dopo una introduzione sui concetti di High-Performance Computing, e dei problemi che si riscontrano nei giorni nostri, sono stati approfonditi nel secondo, i concetti di Power Management e Power Stack analizzando lo stato dell'arte sulla gestione delle potenze, con tutti i componenti ad esso associati. 

Nel terzo capitolo e' stato introdotto il framework DDS che si è voluto analizzare, ed in particolare la sua implementazione \emph{FastDDS} che rappresenta la tecnologia alla base di questa tesi. Si sono inoltre accennati i meccanismi usati in ROS2 per risolvere i problemi analoghi a quello riscontrato nel Power Stack.

Nel quarto, vista la collaborazione con il progetto REGALE, sono stati illustrati i progetti e gli obbiettivi ricercati, introducendo brevemente anche i software open-source che saranno implementati nei vari ruoli del Power Stack. 

Il quinto capitolo è stato utile a presentare in modo esaustivo tutti gli strumenti utilizzati per eseguire i test, oltre agli obbiettivi e al metodo di sperimentazione. Successivamente, nel sesto, sono stati mostrati i risultati degli esperimenti fatti. In primo luogo si è visto l'andamento esponenziale nella fase di discovery risolvibile tramite la sua implementazione di server. Il protocollo più performante si è rilevato essere quello di Shared Memory utilizzabile però solo in presenza di memorie condivise. Al secondo posto UDP e UDP Multicas, ma quest'ultima, solo in presenza di diversi subscriber. Infine non si sono notate particolari differenze nell'usare le wildcards, aprendo le strade alle comunicazioni di tipo gerarchico.
Grazie ai risultati è stato possibile fornire un modello di utilizzo su come è possibile utilizzare e impostare al meglio questo middleware.

In conclusione sono stati mostrati tutte le implementazioni \emph{Dummy} con le relative strutture di comunicazione, prodotte in collaborazione con il progetto REGALE.
%Infine con la imlpementazione dei dummy è stato fornito anche un esempio di utilizzo utile a tutti i possibili interessati nel progetto.
%È stato dimostrato come un framework di comunicazione DDS può essere usato all'interno di un Power-Stack per la gestione di energia in sistemi HPC vincolati dalla potenza al fine di affrontare il problema della limitazione energetica.
% In the second chapter, the phase noise has been studied, focusing on how it is generated and
% on how it can be modelled. Considerations and analysis were made to model the noise atTerahertz frequency as white and Gaussian: this allowed us to model the phase noise as thesuperposition of two components, the Wiener one and the white one.
% In the third chapter, OCDM has been introduced, along with the analytical expression of thereceived signal affected by phase noise. This allowed to distinguish two phase noisecomponents: the Common Phase Error (CPE) and the Inter-Chirp Interference (ICI). A simplecompensation algorithm for CPE, originally introduced for OFDM systems, has beenreported.
% In the fourth chapter, numerical results have been shown. It has been found that the performance of the system, in terms of BER, depends on the number of chirps N. Thed ependence on other parameters was also found, the most significant being the frequencydeviation B: there is an optimum value of B that, for a given SNR, minimizes BER. Thisphenomenon is due to the fact that with a small B, chirps are more compressed and phasenoise easily cause error at the detection, while with a large B, a larger amount of noise (thatis converted into phase noise) affects the system. The trade-off between those two is thereason why an optimum value can be found. This value also depends on N.
% A simple compensation algorithm has been applied to the system: BER has improved, and the magnitude of the improvement depends on B as well. In particular, it is greater when B is nearto the optimum value. It has been found that BER could be reduced to one tenth of the  original value

%TODO: REMOVE %Con questo framework, è possibile specificare e calibrare facilmente l'hardware del sistema. Nel frattempo, è utile anche per compiti come la regolazione delle applicazioni utente per massimizzare le prestazioni o ridurre la domanda di energia. Per verificare la validità e l'utilità del framework, è stato testato con diversi casi di studio. In questi casi di studio, è stata applicata la gestione dell'energia a due applicazioni selezionate e è stata dimostrata la possibilità di costruire e utilizzare un semplice modello di potenza con una relazione lineare tra le prestazioni della CPU e il consumo di energia per derivare il limite di potenza. Questi casi di studio hanno semplicemente dimostrato che il framework può fornire agli utenti un modo semplice per applicare l'ottimizzazione e la gestione dell'energia alle loro applicazioni. Nel futuro lavoro, è previsto valutare il framework proposto con altre politiche/algoritmi di ottimizzazione delle prestazioni e dell'energia e migliorarlo con ulteriori funzionalità come la cooperazione con il software di sistema, i pianificatori di lavori e altri strumenti esterni per arricchire le sue funzionalità.

%TODO: ringraziamenti? 