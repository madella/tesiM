% - Introduzione (contesto: power management, hpc, dds)
%   - problematica 
%   - contributi ( lista della spesa )
%   non eccessivamente lunga
\chapter{Introduzione}
\noindent
In questo ultimo decennio nel contesto dei sistemi informatici si sono manifestate delle sempre crescente richieste di capacità e potenza computazionale mentre le tecnologie ad esse associate si sono avvicinate sempre più ai propri limiti fisici. L'efficienza energetica nei data center è diventata una delle principali preoccupazioni, non solo a causa dei costi monetari, ma anche per la sostenibilità ambientale. Il consumo di elettricità dei data center è in costante aumento, ed è urgente applicare ottimizzazioni hardware e software per ottenere le migliori prestazioni per watt.
% Unito a questo, ci si è trovati di fronte a diverse crisi energetiche che ci ha fatto capire che sia il costo, che l'importanza di utilizzare con più efficienza queste risorse è diventato di vitale importanza.
\subsection{HPC}
\subsection{Power Management}
\subsection{DDS}

% Negli ultimi anni, la rapida crescita dei sistemi di supercalcolo ha portato a una crescente domanda di strategie efficienti per il power management. Poiché il panorama delle capacità omputazionali continua la sua rapida espansione, l'imperativo di soluzioni  sostenibili di gestione dell'energia diventa ancora più evidente. La formidabile potenza computazionale esercitata dai moderni supercomputer è spesso contrastata dal considerevole consumo di energia necessario per alimentare le loro operazioni. Questo fenomeno è evidente anche nell'interesse crescente per il calcolo ecologico, anziché il puro e semplice consumo di energia\cite{Green}\cite{Green2}\cite{hardvard}. In questo contesto, la ricerca di innovative strategie di gestione dell'energia diventa non solo una considerazione economica, ma soprattutto uno sforzo fondamentale per ridurre le conseguenze ecologiche dell'incremento del consumo di energia. Inoltre, le prestazioni degli elementi di calcolo sono intrinsecamente limitate dal consumo di energia, il che implica che migliorare l'efficienza energetica si traduce in una maggiore performance massima.

% Le architetture moderne implementano un controllore termico e energetico integrato nel dado, il cui scopo è fornire prestazioni massime entro limiti fisici ed esternamente imposti. Oltre a ciò, sono stati aggiunti anche altri compiti al controllo energetico e termico, tra cui:
% (i) scambio di messaggi con diversi agenti (come Node-manager, Board Management Controller (BMC) e sistemi operativi) modificando la sua azione di controllo per soddisfare tutti i vincoli dati da questi agenti;
% (ii) essere consapevoli delle implicazioni sulla sicurezza delle sue azioni di controllo, evitando punti operativi fatali a livello di sistema e rilevando e prevenendo attacchi basati sulla sicurezza legati a errori\cite{fbv} o a rumori elettrici virus\cite{pnv}.

% Inoltre, c'è la mancanza di software completi, interoperabili e open-source in grado di gestire e monitorare contemporaneamente il consumo senza influire sull'aspetto più cruciale dei supercomputer: le prestazioni. Attualmente, ci sono diverse utility [\ref{SSEC:runtimes}] capaci di risolvere un dominio di problemi, ma senza la capacità di interagire tra loro in modo diretto.

% L'obiettivo finale sarebbe infatti quello di collegare gli strumenti diversi disponibili utilizzando un approccio distribuito, sfruttando il potenziale del Data Distribution Service (DDS) e del Real-Time Publish-Subscribe (RTPS).% [\ref{SEC:dds}].

% \subsection{DDS \& RTPS} \label{SEC:dds}
% DDS (Data Distribution Service)\cite{DDS} e RTPS (Real-Time Publish-Subscribe)\cite{RTPS} costituiscono due tecnologie fondamentali nel campo delle comunicazioni distribuite e in tempo reale. Queste tecnologie svolgono un ruolo critico nel consentire una trasmissione efficiente e affidabile dei dati tra dispositivi e applicazioni interconnesse, con particolare rilevanza in scenari complessi come i sistemi embedded, l'Internet delle cose e applicazioni ad alte prestazioni come l'HPC.

% In particolare, DDS funge da framework di comunicazione distribuita che facilita lo scambio di dati tra componenti software distribuiti su reti eterogenee e consente di definire la politica di qualità del servizio (QoS). D'altra parte, RTPS funge da protocollo sottostante utilizzato da DDS per realizzare il paradigma di pubblicazione-sottoscrizione all'interno delle reti in tempo reale. RTPS si concentra sulla consegna affidabile di messaggi in tempo reale, garantendo che i dati raggiungano i destinatari appropriati nel modo più efficiente possibile. Questo protocollo gestisce anche aspetti critici come il controllo del flusso dei dati e la sincronizzazione dei nodi.