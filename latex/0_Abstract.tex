\section*{Abstract}
\label{TODO}
Nei sistemi di High-Performance Computing (HPC), la gestione energetica è diventata una delle principali preoccupazioni, non solo a causa dei costi monetari, ma anche per la sostenibilità ambientale e per la progettazione di nuove generazioni di supercomputer\cite{TODO}. Perpendicolarmente all'aumento della potenza computazionale richiesta, le tecnologie associate allo sviluppo dei componenti che stanno alla base dei processori si sono avvicinati sempre più ai loro limiti fisici. E' nato con questo il concetto di Power Management cercando di definire un modello software che ha il compito di gestire la potenza di sistemi di HPC. Data l'eterogeneità di questi sistemi,
nel corso degli anni sono stati proposti sempre più software in grado di funzionare su una specifica configurazione hardware e cercando di risolvere un sottoinsieme limitato di problemi. Lo scopo di questa tesi è di testare un middleware di comunicazione basato su Data Distribution Service(DDS), che faciliterebbe lo scambio di informazioni ad diversi software di Power Management. Questo permetterebbe di creare un power-stack interoperable e con una visione di insieme. Successivamente sarà stilato un modello di utilizzo ed in collaborazione con il progetto REGALE un esempio di implementazione degli attori coinvolti.

La tesi è organizzata nel seguente modo: nel primo capitolo viene introdotto il concetto di Power Management, nel secondo rappresentato lo stato dell'arte. Dopo una introduzione di DDS nel terzo viene presentato il progetto REGALE. Nel quarto verranno riportati i test effettuati con i relativi risultati nel sesto. Dopo una breve introduzione dei prototipi creati sarà presente una conclusione.

