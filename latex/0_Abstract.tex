\section*{Abstract}
Nei sistemi di High-Performance Computing (HPC), la gestione energetica è diventata una delle principali preoccupazioni, non solo a causa dei costi monetari di esercizio, ma anche per la sostenibilità ambientale e per la progettazione di nuove generazioni di supercalcolatori\cite{growth}. Perpendicolarmente all'aumento della potenza computazionale richiesta, le tecnologie associate allo sviluppo dei componenti che stanno alla base dei processori, si sono avvicinati sempre più ai loro limiti fisici. Il concetto di Power Management si propone di gestire la potenza di sistemi di HPC tramite approcci software. Inoltre data l'eterogeneità dei supercalcolatori, nel corso degli anni sono stati proposte soluzioni di Power Management, legate a specifiche configurazioni hardware e con una visione non olistica. %Al giorno d'oggi ci sarebbe la necessità di un software unito, comune ed in grado di operare su diverse tipologie di architetture.
Lo scopo di questa tesi è di: (i)  caratterizzare le prestazioni dei componenti di DDS al fine di creare un modello costi/benefici relativo alle scelte progettuali effettuate; (ii) sviluppare in collaborazione con i partner del progetto EuroHPC JU REGALE un middleware di comunicazione basato su Data Distribution Service (DDS), per facilitare lo scambio di informazioni tra i diversi software di Power Management; (iii) realizzare una libreria di template di attori di power management integrati con il middleware sviluppato.

%Tale middleware è alla base di futuri Power Software Stack interoperabli e con una visione di insieme. Successivamente sarà stilato un modello di utilizzo, ed in collaborazione con il progetto REGALE, un esempio di implementazione degli attori coinvolti.
