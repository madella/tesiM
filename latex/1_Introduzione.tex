% - Introduzione (contesto: power management, hpc, dds)
%   - problematica 
%   - contributi ( lista della spesa )
%   non eccessivamente lunga
\chapter{Introduzione}
\noindent
% HPC ============================================================================
Nei settori come ricerca scientifica, simulazione, l'analisi dei dati, la progettazione e ingegneria, dove la capacità di eseguire analisi dettagliate e computazioni intensive è essenziale per il progresso scientifico e tecnologico, l'Elaborazione ad Alte Prestazioni (HPC, High-Performance Computing) rappresenta una risorsa vitale. Questa forma calcolo si distingue per la sua capacità di sfruttare architetture hardware specializzate e tecnologie software avanzate per affrontare compiti computazionali di elevata complessità in tempi ridotti, spingendo i limiti delle prestazioni e dell'efficienza computazionale.%Boh TP
Gli HPC sono progettati per affrontare sfide che richiedono quantità massicce di calcoli, come simulazioni complesse, analisi dei dati su larga scala e ottimizzazione di progetti. Ovviamente tutto questo ha un costo, e in questo ultimo decennio si sono manifestate delle sempre crescenti richieste di capacità e potenza computazionale mentre le tecnologie ad esse associate si sono avvicinate ai propri limiti fisici. %TODO:cite
La gestione energetica e della potenza di sistemi HPC è diventata una delle principali preoccupazioni, non solo a causa dei costi monetari, ma anche per la progettazione di nuove generazioni %% TODO: cite
di supercomputer e per la sostenibilità ambientale. 
% Dennard ========================================================================
Questo è dovuto in parte anche alla fine delle leggi di Denard e Moore, che avevano previsto una crescita continua della capacità di calcolo e della densità dei transistor nei circuiti integrati % TODO: cite
Tali leggi, che avevano guidato l'industria informatica per decenni, hanno perso progressivamente la loro validità poiché la fine dell'innovazione dei transistor % Spiego il perchè?
ha comportato un consumo energetico crescente al crescere della velocità e capacità computazionale, rendendo sempre più difficilmente il mantenimento di questi sistemi. %Magari cita il fatto che si sono resi necessari i PowerManagement
%Utilizzare efficientemente l'energia fornita e ottimizzare le prestazioni delle applicazioni scientifiche sotto il controllo di potenza ed energia è una sfida per diverse ragioni, tra cui il comportamento dinamico delle fasi, le variazioni nella produzione e l'aumento dell'eterogeneità a livello di sistema. %Paper PowerStack
% Power management ================================================================
Con Power Management si definisce una stack software che gestisce la potenza e l'energia dei sistemi HPC e standardizza le interfacce tra diversi livelli di componenti software. Uno degli aspetti chiave di un PowerStack è definire una visione globale per la gestione energetica, estensibile ed in grado di ottimizzare la metrica di efficienza desiderata, consapevole dell'applicazione, in modo da poter scambiare potenza, energia e tempo di risoluzione al fine di ottimizzare l'efficienza di un'applicazione HPC. 
Il secondo aspetto è definire un'interfaccia standard per interagire con i controlli software e hardware di ottimizzazione su sistemi HPC di diversi fornitori.
%Una limitazione primaria della maggior parte, se non di tutti, di questi sforzi è che la ricerca sull'ottimizzazione è stata limitata esclusivamente ai singoli livelli del PowerStack. Le opportunità di ottenere ulteriori miglioramenti nell'efficienza energetica dalla messa a punto collettiva di due o più livelli del PowerStack sono rimaste in gran parte inesplorate.
% Problema =======================================================================
Mentre sono state proposte diverse tecniche per la gestione della potenza e dell'energia, la maggior parte di queste tecniche è stata ideata per soddisfare singole esigenze di uno specifico centro di calcolo ad alte prestazioni o per obiettivi di ottimizzazione mirati su un insieme di problemi. Un recente studio [22] condotto dal gruppo di lavoro EEHPC [9] ha concluso che la maggior parte di tali tecniche manca di una consapevolezza globale necessaria per ottenere le migliori prestazioni di sistema e throughput. Inoltre, ciascuna tecnica tende a migliorare la gestione di potenza ed energia per un sottoinsieme diverso dell'hardware del sito o del sistema e a diverse granularità (spesso in conflitto). Sfortunatamente, le tecniche esistenti non sono state progettate per coesistere simultaneamente su un unico sito e cooperare nella gestione in modo efficiente. %Traduzione GPT di paper PS, da cambiare
% Per affrontare queste lacune, la comunità HPC ha bisogno di uno stack completo per la gestione di energia e potenza. 
% DDS =======================================================================
L'obiettivo finale sarebbe infatti quello di collegare gli strumenti disponibili utilizzando un approccio distribuito, sfruttando il potenziale del Data Distribution Service (DDS)[\ref{SEC:dds}] e del Real-Time Publish-Subscribe (RTPS).


\section{Contributi}
- Definire le interfacce tra questi livelli per tradurre gli obiettivi a ciascun livello in azioni al livello inferiore adiacente.
- Promuovere l'ottimizzazione end-to-end attraverso diversi livelli del PowerStack.
% Negli ultimi anni, la rapida crescita dei sistemi di supercalcolo ha portato a una crescente domanda di strategie efficienti per il power management. Poiché il panorama delle capacità omputazionali continua la sua rapida espansione, l'imperativo di soluzioni  sostenibili di gestione dell'energia diventa ancora più evidente. La formidabile potenza computazionale esercitata dai moderni supercomputer è spesso contrastata dal considerevole consumo di energia necessario per alimentare le loro operazioni. Questo fenomeno è evidente anche nell'interesse crescente per il calcolo ecologico, anziché il puro e semplice consumo di energia\cite{Green}\cite{Green2}\cite{hardvard}. In questo contesto, la ricerca di innovative strategie di gestione dell'energia diventa non solo una considerazione economica, ma soprattutto uno sforzo fondamentale per ridurre le conseguenze ecologiche dell'incremento del consumo di energia. Inoltre, le prestazioni degli elementi di calcolo sono intrinsecamente limitate dal consumo di energia, il che implica che migliorare l'efficienza energetica si traduce in una maggiore performance massima.

% Le architetture moderne implementano un controllore termico e energetico integrato nel dado, il cui scopo è fornire prestazioni massime entro limiti fisici ed esternamente imposti. Oltre a ciò, sono stati aggiunti anche altri compiti al controllo energetico e termico, tra cui:
% (i) scambio di messaggi con diversi agenti (come Node-manager, Board Management Controller (BMC) e sistemi operativi) modificando la sua azione di controllo per soddisfare tutti i vincoli dati da questi agenti;
% (ii) essere consapevoli delle implicazioni sulla sicurezza delle sue azioni di controllo, evitando punti operativi fatali a livello di sistema e rilevando e prevenendo attacchi basati sulla sicurezza legati a errori\cite{fbv} o a rumori elettrici virus\cite{pnv}.

% Inoltre, c'è la mancanza di software completi, interoperabili e open-source in grado di gestire e monitorare contemporaneamente il consumo senza influire sull'aspetto più cruciale dei supercomputer: le prestazioni. Attualmente, ci sono diverse utility [\ref{SSEC:runtimes}] capaci di risolvere un dominio di problemi, ma senza la capacità di interagire tra loro in modo diretto.

% L'obiettivo finale sarebbe infatti quello di collegare gli strumenti diversi disponibili utilizzando un approccio distribuito, sfruttando il potenziale del Data Distribution Service (DDS) e del Real-Time Publish-Subscribe (RTPS).% [\ref{SEC:dds}].

% \subsection{DDS \& RTPS} \label{SEC:dds}
% DDS (Data Distribution Service)\cite{DDS} e RTPS (Real-Time Publish-Subscribe)\cite{RTPS} costituiscono due tecnologie fondamentali nel campo delle comunicazioni distribuite e in tempo reale. Queste tecnologie svolgono un ruolo critico nel consentire una trasmissione efficiente e affidabile dei dati tra dispositivi e applicazioni interconnesse, con particolare rilevanza in scenari complessi come i sistemi embedded, l'Internet delle cose e applicazioni ad alte prestazioni come l'HPC.

% In particolare, DDS funge da framework di comunicazione distribuita che facilita lo scambio di dati tra componenti software distribuiti su reti eterogenee e consente di definire la politica di qualità del servizio (QoS). D'altra parte, RTPS funge da protocollo sottostante utilizzato da DDS per realizzare il paradigma di pubblicazione-sottoscrizione all'interno delle reti in tempo reale. RTPS si concentra sulla consegna affidabile di messaggi in tempo reale, garantendo che i dati raggiungano i destinatari appropriati nel modo più efficiente possibile. Questo protocollo gestisce anche aspetti critici come il controllo del flusso dei dati e la sincronizzazione dei nodi.