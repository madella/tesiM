% Caratteri : 20.000 disponibili.
%TITLE {Analisi e sviluppo di un runtime per il power management di sistemi HPC basato su DDS}

%Domande da fare:
% Devo/posso citare REGALE ed i suoi compomenti?

\section{Introduction}
% Presentazione del progetto di ricerca: Includi una breve descrizione del tema trattato, mettendo in risalto l’importanza e l’eventuale originalità.

% WHY
% OLD: In recent years, the rapid growth of super-computing systems has led to an increased demand for efficient power management strategies. As these systems continue to evolve, optimizing power consumption has become a critical challenge to address. Effective power management not only contributes to reducing energy costs but also has a positive impact on the environment by minimizing the carbon footprint associated with data centers and large-scale computing infrastructures. Inoltre, lo sviluppo power manager open source permette una maggiore flessibilità e adattabilità alle esigenze specifiche di diversi ambienti HPC. 

% In un sistema di power-management sono presenti diversi tipi di attori (node-manager, job-manager, monitor, etc.) ognuno dei quali deve interagire con centinaia di entità (nodi, cores). Tutto questo va a generare un traffico non indifferente di dati che deve essere gestito real-time in modo efficente e sicuro.  L'area di interesse di questo progetto, è quello di  implementare un servizio DDS tra i tanti disponibili, all'interno di uno stack oper-source di power management per HPC. 

In recent years, the rapid growth of supercomputing systems has led to an increased demand for efficient power management strategies. As these systems continue to evolve, optimizing power consumption has become a critical challenge to address. Effective power management not only contributes to reducing energy costs but also has a positive impact on the environment by minimizing the carbon footprint associated with data centers and large-scale computing infrastructures. Furthermore, the development of open-source power managers allows for greater flexibility and adaptability to the specific needs of different HPC environments.

In a power management system, there are several types of actors (node-manager, job-manager, monitor, etc.) each of which must interact with hundreds of entities (nodes, cores). All this generates a significant amount of data traffic that must be managed in real-time in an efficient and secure manner. The area of interest of this project is to implement a DDS service among the many available within an open-source power management stack for HPC.

\subsection{DDS \& RTPS}

DDS (Data Distribution Service) and RTPS (Real-Time Publish-Subscribe) constitute two pivotal technologies in the realm of distributed and real-time communications. These technologies play a critical role in enabling efficient and reliable data transmission among interconnected devices and applications, holding particular significance in intricate scenarios such as embedded systems, the Internet of Things (IoT), and high-performance applications like High-Performance Computing (HPC).

Specifically, DDS serves as a distributed communication framework that facilitates data exchange among software components distributed across heterogeneous networks. Built upon a publish-subscribe model, DDS establishes a mechanism for efficient data sharing. On the other hand, RTPS serves as the underlying protocol employed by DDS to realize the publish-subscribe paradigm within real-time networks. RTPS focuses on the dependable delivery of real-time messages, ensuring that data reaches the appropriate recipients in the most efficient manner. This protocol manages critical aspects such as data flow control, node synchronization, and quality of service management.

\subsection{Components}

\section{State of the Art} \label{SEC:soa}
% Stato dell’arte: Ricostruisci le teorie e gli studi che riguardano il tema studiato. Individua le ricerche utili alla tua ricerca e identifica le categorie chiave del dibattito sul tema di ricerca2.

To the best of my knowledge, the current landscape of power management solutions for power management strategies are often characterized by proprietary hardware and software implementations.

To the best of my knowledge, la maggior parte del panorama del power management, per le applicazioni di HPC, è affrontata tramite soluzioni hardware e software proprietarie, principalmente sviluppate in modo ad hoc dai diversi centri di calcolo. Le alternative più complete e open-source che coprono l'ambito trattato sono GEOPM \cite{GEOPM} e REGALE \cite{REGALE}. Inoltre, uno studio approfondito sull'implementazione e sull'utilizzo di DDS è rappresentato dal middleware implementato in ROS2 chiamato \emph{rmw\_rtps}.

\subsection{GEOPM}
\subsection{REGALE}
\subsection{Middleware}
% ROS: Uno dei più famosi esempi middleware che implementa DDS come canale di comunicazione è quello usato da ROS2. Infatti rmw




\section{Project’s Description} \label{SEC:pd}
% Definizione del problema e domande di ricerca: Definisci in termini chiari ed immediatamente comprensibili la questione che vuoi indagare. Articola il problema in domande di ricerca a cui vuoi rispondere.

% A main point of the PhD project will be the analysis of the performance of different control designs and structures into answering the power and thermal control problem of current and future processors, with the focus on more comprehensive designs. This includes the investigation of more advanced and newer control structures with respect to the PI control, including also machine and deep learning. An interesting approach is investigating Spiking Neural Network (SNN) which are proved comparable to PID while also being capable to adapt at run-time and running on ultra-low power chips \cite{snn1}, \cite{snn2}.

% A second point of the project will be to try to fill the void in the open-source community regarding an open-source hardware/software power and thermal controller suitable to be integrated on a broad range of different processors. Carrying on a hardware/software co-design can lead to interesting design choices and the possibility to obtain a performing and efficient system. For example to improve parallel computation of some types of control structures (e.g. machine learning, multiple PIDs, Deep-Reinforcement Learning, Model Predictive Control) the hardware of the controller could contain a multicore cluster for parallel acceleration.

% A middle step will be the development of a simulation framework, needed to develop the firmware in co-design with the hardware of the controller and to compare the various control designs.

% The project will focus on High Power Computing (HPC) processors, but the results will be exported and adapted to other types of scenarios, like edge, mobile, automotive, and workstations.

The primary objective of this research is to analyze, design, and develop a runtime system for power management that exploits and leverage the capabilities of the DDS protocol considering all entities that are included and all their interactions in the system. 
By dynamically adjusting the power consumption of individual components based on their current workloads and communication patterns, the proposed system aims to achieve substantial energy savings without compromising system performance.
A second point of the project is to explore the potential of utilizing DDS as a basis for runtime power management. By leveraging the advantages of DDS, we intend to develop an open-source power management solution that offers transparency, adaptability, and scalability. %TOTRANSALTE 
Si proverà anche a fornire un middleware per poter scegliere a piacere una implementazione diversa di DDS


\section{Expected Results}
% Obiettivi ed ipotesi della ricerca: Identifica lo scopo della tua ricerca e definisci le finalità conoscitive delle indagine. Esplicita come intendi rispondere alle domande di ricerca2.
This project is expected to achieve three main contributions:
\begin{itemize}
    \item the analysis of several advanced control structures aimed at answering the new upcoming requirements and constraints originated from the increasing complexity of processors in the edge and high-performance computing domains;
    \item The exploration of more compute-intensive control paradigm (model predictive control, deep reinforcement learning), and HW architectures to execute them efficiently.
    \item the development of a simulation framework for the co-design of all the parts of the control system (HW, run-time/RTOS, control policy, interfaces with processors and sensors);
    \item the development of an open-source power controller firmware based on open-source hardware, aimed at being implemented in a broad range of chips.
\end{itemize}


\section{Proposed project timeline}
% Cronoprogramma e fattibilità: Includi un cronoprogramma per la realizzazione del tuo progetto di ricerca e discuti la fattibilità del tuo approccio.
\begin{itemize}
\item{Year 1:
\begin{itemize}
    \item Literature overview on advanced control designs and on the State of the Art control algorithms applicable to the power and thermal control of a processor.
    \item Creation of a simulation framework. Development of system and controller models.
    \item Initial Firmware architecture design.
\end{itemize}
}

\item{Year 2:
\begin{itemize}
    \item Analysis and comparison of the identified control structure designs.
    \item Intermediate version of the controller Firmware targeting an identified open-source hardware design.
\end{itemize}
}

\item{Year 3:
\begin{itemize}
    \item SECURITY? 
    \item Adapt the project to a broad range of scenarios and possible implementations.
\end{itemize}
}
\end{itemize}



\section{Outline of the proposed findings assessment criteria}
The criteria to asses the proposed findings will be:
\begin{itemize}
    \item an open-source publication on GitHub of an operational firmware for a processor power controller, associated with an open-source RTL component regarding the controller hardware.
    \item the possibility to consider the implementation of the open-source controller project in a real processor design.
    \item an exhaustive analysis of the control structures relevant to the power and thermal control of a processor, publishable on a paper.
    \item the development of a simulation framework able to provide reliable results.
\end{itemize}


% Referenze bibliografiche: Includi una lista di riferimenti bibliografici per le fonti citate nel tuo progetto di ricerca.