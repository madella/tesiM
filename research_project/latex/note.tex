% Cirte security part in introduction?
The primary objective of this research would be the design and implement an advanced middleware solution that promotes energy-efficient computing while ensuring robust measures. The proposed middleware will facilitate the seamless integration of HPC resources distributed across different nodes, enabling efficient resource utilization and minimizing power consumption. By leveraging the capabilities of RTPS and the FastDDS stack, the middleware will provide reliable and real-time communication channels among HPC components while optimizing resource allocation to achieve high performance. Furthermore, the proposed middleware will focus on promoting green computing practices by optimizing energy consumption. It will employ intelligent resource management techniques, such as dynamic workload balancing and power-aware scheduling algorithms, to minimize energy waste and maximize the utilization of available resources. The middleware will also provide monitoring and profiling capabilities to analyze energy consumption patterns and identify optimization opportunities. To evaluate the effectiveness of the proposed middleware, a comprehensive experimental setup will be established. Performance benchmarks will be conducted using representative HPC workloads, and energy consumption metrics will be measured. The results will be compared against existing HPC systems to quantify the energy savings achieved and validate the efficiency and scalability of the proposed solution. In summary, this PhD proposal aims to develop a secure and distributed middleware for green computing in the context of HPC. By leveraging the RTPS protocol and the FastDDS stack, the proposed middleware will enable efficient resource utilization, real-time communication, and enhanced security in HPC systems. The research outcomes will contribute to the advancement of sustainable and energy-efficient computing practices, ensuring a greener future for HPC applications.

% Spiegare la parte dei vari componenti presenti nel sistema (Derivable 3.2)

(jobmanager) COUNTDOWN: is an open-source runtime library that is able to identify and automatically
reduce the power consumption of the computing elements during communication and
synchronisation of MPI-based applications. COUNTDOWN saves energy without imposing a
significant performance penalty by lowering CPUs power consumption only during waiting
times for which performance state transition overheads are negligible.

EXAMON: (Exascale Monitoring) is a lightweight monitoring framework for supporting
accurate monitoring of power/energy/thermal and architectural parameters in distributed
and large-scale high-performance computing installations. EXAMON is composed of different
layers, each of them with multiple components. The integration of different data sources is
handled by the compositional nature of the infrastructure, where new components can be
added seamlessly provided that they respect the correct data formats.


(Nodemanager) EAR: is a job accounting and power monitoring software focused on power and application
performance. EAR collects metrics in two granularities: loop and application. Loop
corresponds to a piece of code executed in a repetitive way. 

OAR

BEO Bull Energy Optimizer (BEO) is intended to be integrated with OAR through a prologue and an
epilogue, which implementation depends on the design of the Application-Aware Power Capping (AAPC) mechanism. The architecture of this integration is presented in the first
sub-sectio

OCDB

Maybe scheme?

% Spiegare come fastRTPS potrebbe essere utile alla interconnessione tra i vari componenti

% Spiegare il migliore approccio possibile per DDS
%   Discovery
%   partitions
%   funzionamento in generale

% Conclusioni
